\section{Introduction}\label{sec:intro}
Bangladesh e-Passport Portal \footnote{\url{https://www.epassport.gov.bd/landing}}, the subject web application for our study, is one of the most notable publicly available software projects in Bangladesh. The portal is currently being used to serve the citizens of the country, offering both registration and renewal of an electronic passport (e-passport). The e-passport portal was introduced as a part of the E-passport and Automated Border Control Management project, jointly designed and developed by the German identity service provider company Veridos GmBH, and the Department of Passport and Immigration of the People’s republic of Bangladesh with assistance from the Bangladesh Army and the Security Services Division of the Home Ministry \cite{c21}. The project, in its all, was implemented at an approximate cost of 4569 crore taka. The online application portal enjoyed its pilot deployment phase starting in January 2020 \footnote{\url{http://www.dip.gov.bd/site/view/innovation/Piloting\%20Project}}, and finally was adopted country-wide in the middle of the same calendar year. 

The services provided by the portal can be categorized into 2 major sections. First comes the online application that the user has to fill up completely to apply for a passport. It also contains a status-checking dashboard to track and notify about the progress of passport processing. It contains an urgent application section to serve emergency cases. Moreover, the system incorporated an online payment platform to ease the fee payment process. Second, comes the assisting segment of the website. It is presented with the information section consisting of the step by step instructions, payment options, frequently asked questions (FAQs), notices, and last but not the least, a user feedback system.

Being a significant example of one the most used public software projects in the country, the Bangladesh e-Passport Portal serves a massive amount of people every year, with a very high daily usage time and user load. Hence, we chose this website to analyze its functionality and performance from the eyes of a regular user. We collected user experience data to validate our analysis and to find new insights about the strong and weak features of the website as well.
